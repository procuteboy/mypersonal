\documentclass[12pt]{article}
%\documentclass[prd, showpacs,nofootinbib,amsmath,amssymb]{revtex4}
\usepackage{amsfonts, amssymb, amsmath, graphicx, comment, bm, slashed}
\usepackage[colorlinks]{hyperref}
\usepackage{dcolumn}   %from Tomoki
\usepackage{bm}        %from Tomoki
\usepackage{subfig}
\usepackage{caption}
\usepackage{multirow}
%\usepackage[hypertex]{hyperref}   %from Tomoki
\usepackage{mathrsfs}
%%%%%
%\usepackage[dvips]{graphicx}
%\topmargin -0.0in\oddsidemargin -0.01in \textheight 22cm \textwidth
%15.5cm \pagestyle{plain} \baselineskip 30pt

\begin{document}
\vspace{0.5cm}
\begin{center}
	\Large \bf  Higgs condensate and vortices from Ginzburg-Landau Lagrangian\\
	\Large\bf   with magnetic-field-dependent coefficient
\end{center}

\vspace{1cm}
\centerline{ Fu-Ping Peng$^{1}$, Xiao-Bing Zhang$^{1}$ and Yi Zhang$^{2}$}
%\affiliation
\centerline{\small $^{1}$ School of Physics, Nankai University, Tianjin  300071, China}
\centerline{\small $^{2}$ Department of Physics, Shanghai Normal University, Shanghai 200230, China}
%\vspace{8pt}
%\footnotemark{0}
\footnotetext{* zhangxb@nankai.edu.cn (Xiao-Bing Zhang) } \vspace{0.5cm}

\begin{abstract}\rm \noindent
Within a generic Ginzburg-Landau framework, we study the influence of rotated electromagnetic field in the scope of color-flavor-locked-type quark matter. The Higgs modes and their responses to an applied magnetic background are described by using the effective Lagrangian with a medium dependent coefficient. We demonstrate that the charged-Higgs-mode condensate could emerge as  color-flavor unlocked excitations. Its realization as inhomogeneous vortex solutions is also considered. We construct a superfluid-like vortex (string) and then investigate the theoretical possibility of a topologically and energetically stable vorton.
\end{abstract}
%\pacs{11.10.Qc, 12.38.Aw,  25.75.Nq}
% 12.39.Fe,, 12.38.-t}

\baselineskip 12pt

\section{\bf Introduction}

%\vspace{0.2cm}

Strongly interacting matter under the influence of magnetic fields has attracted intensive interests
in recent years, see, e.g., ~\cite{andersen2016phase,kharzeev2013strongly,miransky2015quantum}.
In heavy ion collision experiments, a strong magnetic field up to $eB \sim 2m_{\pi}^2$ with $m_{\pi}$
the pion mass, i.e., $B \sim 10^{18}\text{G}$ can be produced in non-central collisions at Relativistic Heavy
Ion Collider (RHIC), and $B \sim 10^{20}\text{G}$ at Large Hadron Collider (LHC)~\cite{kharzeev2008,skokov2009}.
For astrophysics objects in the universe, the magnetic fields as large as
$B \sim 10^{14}$ - $10^{15}\text{G}$ exist on the surface of Magnetars. While the strength
can reach $B \sim 10^{18}\text{G}$ in the interior of regular neutron stars, a theoretical upper limit to the magnetic field may stand as high as $B \sim 10^{20}\text{G}$ inside self-bound compact stars~\cite{dong2001,lai1991cold}.
%%%%%%%%%%%%%%
Under such circumstances of strong magnetic fields,
answer to the question of how the Quantum chromodynamics (QCD) phase structure can be modified remains
one of the major theoretical challenges.
%%%%%%%%%%%%%%%%%
Up to date, the mapping of QCD phase diagram onto the temperature and
magnetic field plane has not been firmly established.
%%%

Further, through intensive studies in past decades, it is realized that there exists so-called color
superconducting phase of quark matter at high density regime (see, e.g.
Refs.~\cite{alford2004dense,buballa2005njl}). As the most typical color superconductor, the color-flavor-locked (CFL) phase is widely believed to be the ground state of three-flavor QCD at extremely high density and low temperature~\cite{alford1998qcd}. Due to the diquark condensates formed in the CFL phase, the symmetry breaking pattern is
\begin{eqnarray}
G=SU(3)_{C}\times SU(3)_{L}
\times SU(3)_{R}\times U(1)_{B} \rightarrow H=SU(3)_{C+L+R}\equiv SU(3)_{C+F},\label{cfl}
\end{eqnarray}
where the approximate symmetry $U(1)_{A}$ in $G$ and the discrete symmetries in $H$ have been ignored.
It means that an original QCD symmetry, including color, flavor (left- and right-hand) and baryon number
symmetries, is broken down to the color-flavor locked symmetry.
In the scope of color-flavor-locked-type matter, it is of
great interest to investigate which modifications are induced by the magnetic-field background.
%%%%%%%
Besides its theoretical implication to the QCD phase diagram, this topic might be very important to understand the physics of
magnetars since such kind of astrophysics objects are usually suggested to have color-superconducting cores.
%%%%%%%%%%%%%%%%%
%%%%%%%%%%%%%%%%%%%%%%%%
%%%%%%%%%%%%
The main feature relevant for magnetic effect is the rotated electromagnetic mechanism.
By introducing the rotated electric charge that defined by
$\widetilde{Q}=Q_{F}\times {1}_{C}-{1}_{F}\times Q_{C}$ in the color-flavor space, the
$U(1)_{\widetilde{Q}}$ symmetry remains unbroken
and this group is embedded in the color-flavor locked subgroup~\cite{alford1998qcd}.
The diquark condensates in the CFL phase are always neutral in the sense of rotated charge.
There is no Meissner effect so that the electromagnetic
fields propagate inside the color-flavor-locked matter freely~\cite{alford1998qcd,alford2000magnetic}.
Also, the rotated electromagnetic field is made up mostly of a usual
electromagnetic field as long as the color gauge coupling is strong relatively.
%%%
As the consequence, the rotated magnetic field could be described by an external background field
$B$ approximately.
%%%%%%%%%%%%%%%%%%%%%%%%%%%%%%%%%
In the presence of non-vanishing field $B$, there should exist considerable changes in the properties
of color-flavor-locked-type quark matter.
In literature this topic has been widely investigated within the phenomenological Nambu-Jona Lasinio (NJL)
models with four-quark
interactions \cite{ferrer2005magnetic,fukushima2008color,ferrer2006color,ferrer2007magnetic,sen2015anisotropic}.
Due to the response of rotated-charged quarks to $B$, these studies were mainly devoted to the magnetic-field
dependence of color-superconducting gaps. The resulting phase of color-flavor-locked matter is called as the
magnetic color-flavor locked phase.
%%%%%%%%%%%%%%%%%%%%%%
%%%%%%%%%%%%%%%%%%%%%%%%%%%

In present work, we investigate the influences of rotated magnetic field within a generic Ginzburg-Landau (GL)
framework. Such a model-independent method as GL model has been applied to study color-superconducting phase
of dense QCD and the resulting topological vortices for years, see, e.g., ~\cite{giannakis2002ginzburg,iida2002superfluid,balachandran2006semisuperfluid,nakano2008non,eto2014vortices,zhang2015magnetic}.
%%%%%%%%
%%%%%%%%%%%%%%%%%%
To account for color-flavor-locked matter, the GL Lagrangian is developed from the symmetry breaking pattern Eq.~(\ref{cfl}).
In particular, the order parameter describing diquark condensates can be denoted by a complex $3\times3$ matrix.
It is rather simple with respect to the case of a $72 \times 72$ matrix in NJL models. Within the GL framework
the relevant degrees of freedom (dof) are the Higgs modes, i.e., fluctuations of diquark condensates, rather than
the quark dof in a microscopic model.
Our goal in this paper is to explore the magnetic response of rotated-charged Higgs modes.
For the first time, we suggest that the charged Higgs-mode condensate emerges in the presence of an applied field $B$ and originates
from the excitations of charged, color-flavor-unlocked diquark condensates. This issue has not been considered in the NJL
models and it might shed light on unknown aspects of magnetic effects on color-flavor-locked-type matter.
%%%%%%%%%
In addition to clarifying the Higgs condensate and its dynamics via a qualitative GL analysis, we are concerned with the
possible realizations as inhomogeneous vortex solutions. From the properties of CFL vortices as well as the symmetry
consideration, the Higgs-condensate emergence is attributed to a simple $U(1)$ dynamical breaking preliminarily.
As a consequence, we construct the superfluid-like vortex solution for Higgs condensate which is very similar as
a $U(1)_B$ vortex consisting of the usual color-flavor-locked matter.
%%%%%%%%%
Moreover, the theoretical possibility of topologically- and energetically-stable vortons is pointed out in the
situation with both color-flavor-unlocked Higgs condensate and color-flavor-locked diquark condensate.
%%%%%%%%%%%%%

This paper is structured as follows. After a brief review of the original GL Lagrangian accounting for CFL, in
Sec.~\ref{sec:2}, particular emphasis are placed on the magnetic response of charged Higgs modes and its
consequences on the formalism. In Sec.~\ref{sec:3}, we study the formations of superfluid-like vortices and vortons
with different boundary conditions. Sec.~\ref{sec:4} is devoted to discussions of some open problems.

\section{\bf Formalisms}
\label{sec:2}
\vspace{0.2cm}

For the three-flavor color-superconducting quark matter, the GL order parameter is usually denoted by a complex $3\times3$ matrix $\Phi$. With the flavor indices $i = 1, 2, 3 = u, d, s$ and the color indices $\alpha = 1, 2, 3 = r, g, b$, the matrix element $\Phi_{i \alpha}$ account for the possible condensates of quarks with non-$\alpha$ colors and non-$i$ flavors and it can be expressed explicitly as follows
\begin{equation}
  \label{eq:diquarkmatrix}
  \Phi =
  \begin{pmatrix}
    \Phi_{gb}^{ds} &  \Phi_{gb}^{su} & \Phi_{gb}^{ud} \\
    \Phi_{br}^{ds} &  \Phi_{br}^{su} & \Phi_{br}^{ud} \\
    \Phi_{rg}^{ds} &  \Phi_{rg}^{su} & \Phi_{rg}^{ud}
  \end{pmatrix}.
\end{equation}
In the CFL phase, the diquark condensates correspond to the pairing of quarks in the flavor and color anti-triplet channel~\cite{alford1998qcd}.
%
Thus, only the diagonal elements of $\Phi$ is allowed for CFL and are well studied in~\cite{giannakis2002ginzburg,iida2002superfluid}.
Except for such kind of color-flavor-locked species, the non-diagonal elements that belong to color-flavor-unlocked species
will be explored in later sections of current paper.

The GL approach is purely based on symmetry consideration and its Lagrangian is developed from the symmetry breaking pattern Eq.~(\ref{cfl}).
%%%%%%%%%%%%
To quartic order in $\Phi$, the GL Lagrangian is written as
\begin{equation}
\mathcal{L}= \frac{1}{4} (f_{\mu\nu})^2 +
\text{Tr}\left[\kappa_3(\vec{\mathcal{D}}\Phi)^\dagger\vec{\mathcal{D}}\Phi
  -\alpha\Phi^\dagger\Phi -\beta_2(\Phi^\dagger\Phi)^2\right]
-\beta_1(\text{Tr}[\Phi^\dagger\Phi])^2 +\cdots .\label{gl}
\end{equation}
where the coefficients $\alpha$, $\beta_1$ and $\beta_2$ are used to characterize the color-flavor-locked-type matter and $\kappa_3$ is introduced for the generality of GL theory.
%%%
Also, a rotated electromagnetic field is introduced via the field strength tensor $f_{\mu\nu}$
as well as the covariant derivative ${\mathcal{D}}$.
%%
For the static electromagnetic field $\vec{A}$,
%%
the covariant derivative acting on $\Phi$ might be given by
\begin{equation}\label{eq:coderi}
\vec{\mathcal{D}}\Phi =\vec{\nabla} \Phi -ie\vec{A} T_{\widetilde{Q}}\Phi,
\end{equation}
where the unit of rotated charge has been denoted as an electron charge $e$ approximately.
Keeping in mind that the gauge field originate from an unbroken $U(1)_{\widetilde{Q}}$ group, in particular,
$T_{\widetilde{Q}}$ in Eq.~(\ref{eq:coderi}) is the rotated-charge operator defined in the color-flavor space. In this sense, $T_{\widetilde{Q}}$ does not correspond the matrix
$\text{diag}(-2/3,1/3,1/3)$ that defined for usual charges and its form needs to be elaborated
according to the concrete rotated-charge properties of $\Phi_{i \alpha}$.
%%%%%

\vspace{0.2cm}
\textbf{A. Higgs modes and their masses in CFL}
\vspace{0.2cm}

For the most-symmetric CFL phase, the GL order parameter reads
\begin{equation}
  \label{eq:phi}
  \Phi =
  \begin{pmatrix}
    d & 0 & 0 \\
    0 & d & 0 \\
    0 & 0 & d
    \end{pmatrix},
\end{equation}
where color-flavor-locked condensates are assumed to be equal in an ideal situation with massless quarks.
%%
Since $d$ is always neutral in the sense of rotated charge, the operator $T_{\widetilde{Q}}$ as well as
the covariant derivative have no effect on such order parameter.
At least within the GL framework, the presence of magnetic fields
does not influence the CFL phase with Eq.~\eqref{eq:phi}.
%%%%%%
In this sense, the generic Lagrangian Eq.~\eqref{gl} can recover to an usual Lagrangian without
introducing the rotated electromagnetic field.
As well known, the color-flavor-locked condensate $d$ acquires its vacuum expectation value (VEV) as long as
the coefficient $\alpha$ is negative whereas $\beta_1$ and $\beta_2$ are positive.
%%in Eq.~\eqref{gl}
The resulting VEV corresponds to an uniform color-superconducting gap and its value $v$ is obtained by
\begin{equation}
  \label{eq:dvaccum}
v^2 = -\frac{\alpha}{3\beta_1+\beta_2}.
\end{equation}
%%%%%%%%%
%\begin{equation}  \text{VEV}(\Phi)=\text{diag}(v,v,v) ,\label{cflground}\end{equation}
%\cite{iida2002superfluid}.
% Then the GL potential can be also rewritten as a Mexican-hat-shape form
% \begin{equation}
% \label{eq:dpotential}
% \mathcal{V}_d \sim - \alpha(d^2 - v^2)^2.
% \end{equation}
Obviously, it is a standard GL strategy where the negative $\alpha$ determines color-flavor-locked condensate mainly.

When the CFL vacuum is given, Higgs modes appear as fluctuations of diquark condensate.
Due to the symmetry breaking pattern Eq.~(\ref{cfl}), the Higgs modes can be parameterized in the space $G/H \simeq U(3)$.
Thus, we introduce a Higgs singlet $\phi$ and Higgs octet $\zeta^a$ ($a = 1, 2, \cdots, 8$) by perturbing the diquark-condensate matrix
\begin{eqnarray}
\Phi=v\textbf{1}_3+\frac{\phi+i\varphi}{\sqrt{2}}\textbf{1}_3+\frac{\zeta^a+i\chi^a}{\sqrt{2}}T^a,
\label{pert}
\end{eqnarray}
where $T^a$ is the generators of $U(3)$ and either $\varphi$ or $\chi^a$ is of the pseudo Nambu-Goldstone mode.
%%with $\text{Tr}[T^a T^b]=\delta^{ab}$.
%In Eq.~(\ref{pert}), the singlet field $\varphi$ and the octet fields $\chi^a$ correspond one to
%one to the Higgs fields $\phi$ and $\zeta^a$, respectively. They belong to the pseudo Nambu-Goldstone modes.
It is important to realize that the Higgs mode $\phi$ is associated with the $U(1)_B$ symmetry and $\zeta$ is associated
with the chiral symmetry. 
%%%%%%%%%%%%%%%%
Each of spontaneous symmetry breaking is responsible for the existence of nonzero Higgs mass.
Around the CFL vacuum, these Higgs masses are easily found to be
\begin{eqnarray}
m_\phi^2={-2\alpha}/{k_3},\\ m_\zeta^2={4\beta_2 v^2}/{\kappa_3}	,\label{mhiggs}
\end{eqnarray}
respectively.
%%%m_\phi^2=\frac{-2\alpha}{k_3},\\ m_\zeta^2=\frac{4\beta_2}{\kappa_3}	v^2.\label{mhiggs}

In GL formalism the Higgs modes are the basic dof. In terms of the Higgs masses,
the interacting potential can be rewritten as
\begin{eqnarray}
\mathcal{V}_\phi (\Phi)=
\kappa_3\frac{m_\phi^2}{12v^2}(\text{Tr}[\Phi^\dagger\Phi-v^2])^2,
\label{glmphi}
\end{eqnarray}
and
\begin{eqnarray}
\mathcal{V}_\zeta (\Phi)=
\kappa_3\frac{m_\zeta^2}{4v^2}\text{Tr}\left[\left<\Phi^\dagger\Phi\right>^2\right],
\label{glzeta}
\end{eqnarray}
where the definition $\left<M\right>\equiv M-(1/N)\text{Tr}M$ is used for a $N\times N$ matrix $M$.
%%  spatial variations of $\Phi$ does not influence  and
%%the spatial variations of $\Phi$
%%as well as the stiffness coefficient $\kappa_3$  yet.
The advantage of Eqs.(\ref{glmphi}) and (\ref{glzeta})
%over the original formalism
is that the GL potential decompose into two parts by utilizing two kinds of Higgs modes. The part $\mathcal{V}_\phi$
stems from a $U(1)_B$ breaking and accounts for the trace contribution. As defined in Eq.~(\ref{glzeta}), all the
traceless contribution is encoded in $\mathcal{V}_\zeta$.
%%, 
Also, the traceless potential $\mathcal{V}_\zeta$ indicates the non-Abelian feature of Eq.~(\ref{cfl}), i.e. $SU(3)_{C}\times SU(3)_{L}\times SU(3)_{R}$ is broken to the color-flavor-locked symmetry $SU(3)_{C+L+R}$.
%%

The potential with Higgs singlet $\mathcal{V}_\phi$ is sufficient for describing the CFL free energy. Because of
the vanishing traceless contribution from Eq.~\eqref{eq:phi}, $\mathcal{V}_\zeta$ has actually no effect on
color-flavor-locked species. However, this is not the case for the
color-flavor-unlocked species that originally prohibited in CFL. If there exist some factors (either internal or external) that break the chiral symmetry, the unlocked species would have the opportunity of being excited
and the potential $\mathcal{V}_\zeta$ would play the crucial role.
%%%%%%%%%
This point can be understood more clearly from the other angle as follows.
In principle, the GL order parameter should be given by the matrices $\Phi_L$ and $\Phi_R$, which transform under
the actions of left- and right-hand symmetries respectively.
%%%%%%%%%
Although $\Phi=\Phi_R=\Phi_L $ has been adopted in Eq.~(\ref{gl}), the interacting term up to quartic order can be
decomposed into two parts as well. While the terms with $\alpha$ and $\beta_1$ are made up of $\Phi_L^\dagger\Phi_L$
and $\Phi_R^\dagger\Phi_R$, the $\beta_2$ term involves the products $\Phi_R^\dagger\Phi_L$ and $\Phi_L^\dagger\Phi_R$
relevant for the violation of $\Phi_L=\Phi_R$.
For current consideration, the applied magnetic field plays the role of above-mentioned
external factor. It causes the orientation of $\Phi_L$ different from that of $\Phi_R$ and makes the chiral symmetry (exactly, the color-flavor-locked symmetry) be perturbed.
In this case it is the $\beta_2$ term, equivalently the traceless potential with Higgs octet, be much more important
than the other term/potential, as will be seen later.
%%%%%%%%%%%%%%%%%%%%%%%

\vspace{0.2cm} \textbf{B.  Magnetic-dependent Higgs mass and Higgs condensate}  \vspace{0.2cm}
%%\vspace{0.2cm} \textbf{B.  An effective potential with in-medium mass }  \vspace{0.2cm}

%%%%%%%%%
In this subsection, we consider the rotated-charge properties of matrix elements of $\Phi$ and investigate the
possible response to the magnetic field.
%%%%%%%%
%%%%%%%%%%%%%%%%%%%%%
As color-flavor-locked species, the diagonal element $d$ is always neutral and the Higgs modes are also neutral.
Since these modes do not respond to a magnetic field directly, their masses are given by $m_\phi$ and $m_\zeta$
of course.
Now we may ask what happens for non-diagonal elements. According to Eq.(\ref{eq:diquarkmatrix}), it is easily
observed that part of these elements are charged.
In the convention of $\widetilde{Q}$ in color-flavor space, the rotated charges of $\Phi_{gb}^{su}$,
$\Phi_{gb}^{ud}$, $\Phi_{br}^{ds}$ and $\Phi_{rg}^{ds}$ are $+1,+1,-1,-1$ respectively whereas the other elements
are neutral ( see e.g., Table~1 in Ref.\cite{zhang2015magnetic}).
The four of charged elements could be better illustrated in the
language of Higgs modes. From the defination Eq.(\ref{pert}), it is found that they correspond to the recombination
of Higgs octet $\zeta^a$, say $\zeta^1 \pm i\zeta^2$ and/or $\zeta^4 \pm i\zeta^5$ respectively.
It tells us that they are nothing but the charged fluctuations of diquark condensate, i.e. the charged Higgs modes.
Although $\zeta^a$ are neutral, their combination might carry non-vanishing charge and thus respond to the magnetic background unavoidably.
Just in this sense, the charged Higgs modes are expected to become excited around the CFL vacuum.
%%%%%
If neglecting the flavor asymmetry, it is practical to describe such Higgs excitations by a
unique field $\delta$ and the order parameter reads
\begin{equation}
  \label{eq:diquarkmatrix0}
\Phi = \begin{pmatrix} v & \delta & \delta \\ \delta^* &  v & 0 \\ \delta^* & 0 & v\end{pmatrix},
\end{equation}
where the vacuum $v$ has been assigned for color-flavor-locked species.

%%%%%xiugaibufen
Without loss generality, let us first apply Eq.~\eqref{eq:diquarkmatrix} to
derive the response of charged Higgs modes. Assuming that the unscreened
magnetic field $B$ is taken along the third spatial direction (the longitudinal
direction), the rotated electromagnetic field makes sense in the transverse
direction. The Landau gauge $\vec{A} = (0,xB,0)$ is chosen and the field 
strength is $f_{12} =B$. Besides the trivial contribution $\frac{1}{2}B^2$,
we focus on the gradient-energy contribution from the covariant derivative.
In the rest frame, the gradient energy density in Eq.~\eqref{gl} reads
\begin{equation}
 v_\perp^2\text{Tr}(\mathcal{D}_\perp \Phi)^\dagger(\mathcal{D}_\perp \Phi)
+ v_\parallel^2\text{Tr}(\mathcal{D}_\parallel\Phi)^\dagger(\mathcal{D}_\parallel \Phi),
\end{equation}
where $v_\perp$ and $v_\parallel$ are the transverse and longitudinal
velocities respectively. Also it is noticed that, for the matrix Eq.~\eqref{eq:diquarkmatrix},
the rotated-charge operator $T_{\widetilde{Q}}$ may be given by
\begin{equation}
  T_{\widetilde{Q}} = \begin{pmatrix}
              &    +1  & +1\\
      -1 &            &      \\
      -1 &            &
  \end{pmatrix},
  \end{equation}
in the formal sense.
% showing the expected separation between the longitudinal $D_\parallel$
% and the transverse $D_\perp$ (to the direction of our concerned magnetic field).
% Also we have ignored the const contribution of applied field $\frac{1}{2}B^2$ from $\frac{1}{4}(f_{\mu\nu})^2$,
% since we are more interested in the dynamics of charged Higgs excitations.
% In order to show the derivation result clearly, we write the covariant derivative on the transverse direction in particular
% \begin{eqnarray}
% \mathcal{D}_\perp \Phi =\partial_x \Phi
% + \partial_y \Phi
% + ie xB T_{\widetilde{Q}}\Phi.
% \end{eqnarray}
By considering the static field $\vec{A}$ and the covariant derivative
$\mathcal{D}_\perp \Phi = \partial_\perp \Phi - iexBT_{\widetilde{Q}}\Phi$
in the transverse direction, we obtain 
the kinetic contributions in the transverse direction
\begin{equation}
   v_\perp^2\text{Tr}(\mathcal{D}_\perp \Phi)^\dagger(\mathcal{D}_\perp \Phi)
  = 4|\partial_x \delta
  + \partial_y \delta
  + ie xB \delta|^2
\end{equation}

In order to derive the magnetic responses for the charged Higgs fields, 
a generalization of the method described in~\cite{Ritus1985Quantum} for arbitrary
charges is required.
% to work in the 
% momentum space, i.e., $\delta(\vec{r}) \rightarrow \delta(k)$.
This method is originally developed
by Ritus in order to determine the Green’s function
of charged fermions in the presence of background magnetic field and then  extended to charged vector fields~\cite{Elizalde2004Neutrino}.
In this method, the charged fields are transformed  to momentum space by using the Landau
quantized wave functions $S_k(\vec{r})$ ~\cite{ferrer2007magnetic,Elizalde2004Neutrino}.
% Using the Landau quantized wave function $S_k(\vec{r})$ described in Refs.~\cite{ferrer2007magnetic,Elizalde2004Neutrino},
For the charged Higgs field $\delta(\vec{r})$, we have
\begin{equation}
\label{eq:transform}
\delta(k) = \int d^3r S_k(\vec{r})\delta(\vec{r}),
\end{equation}
where
\begin{equation}
  S_k(\vec{r}) \sim \text{exp}(ik_y y + ik_z z)D_n(\sqrt{2eB}(x - k_y/eB)),
\end{equation}
with $D_n(x)$ the parabolic cylinder function of degree $n$. And the number $n$ denotes the Landau levels.
Using the transform functions in Eq.~\eqref{eq:transform}, we can show that
\begin{equation}
  |\partial_x \delta + \partial_y \delta + ie xB\delta|^2 \rightarrow
   |   \sqrt{(2n+1) eB} |^2 \delta(k)^2
\end{equation}
After performing the inverse transform again,
 the gradient energy density on the transverse direction becomes $4 v_\perp^2(2n+1) eB\delta^2$.
In the lowest Landau levels approximation, total of the gradient energy arrive at 
 \begin{equation}
\label{eq:derive}
\text{Tr}\left[(\vec{\mathcal{D}}\Phi)^\dagger\vec{\mathcal{D}}\Phi \right] = 
4 v_\parallel^2|\partial_z \delta|^2
- 4 v_\perp^2 eB\delta^2.
\end{equation}
%%\text{Tr}\left[(\vec{\mathcal{D}}\Phi)^\dagger(\vec{\mathcal{D}}\Phi) \right] =4 \left[\partial \delta \partial \delta^* -v_\perp^2 eB\delta^2\right].
%%%%%
In Eq.~\eqref{eq:derive}, the additional $\delta^2$ term being proportional to $eB$ appears.
With respect to the original quadratic term of $\delta$, it
leads to that there exists a change in the mass for charged Higgs modes, which already reported
in detail in our previous work~\cite{zhang2015magnetic}.
There, by considering magnetic-field influences on the spectrum
of Higgs octet, the squared mass for charged modes was found to be
\begin{equation}
\label{eq:magneticmass}
(m_\zeta^{eff})^2 = m_\zeta^2 - v_\perp^2eB,
\end{equation}
at the leading order of $eB$. Obviously, the result is equivalent to that obtained from Eq.~\eqref{eq:derive}.
%%
Physically speaking, it means the degenerated masses of charged and neutral
Higgs modes are dismissed.
In other words, the charged modes respond to an applied field $B$ directly, whereas the neutral modes do not.
If a relative rotation of the $\Phi_L$ and $\Phi_R$ orientations take place,
as above stressed, the magnetic response would be attributed to changes in the traceless potential.
%%

Hence it is reasonable to take a replacement $m_\zeta^2 \rightarrow (m_\zeta^{eff})^2$, and this yields
an effective form of the traceless potential
\begin{equation}
 \mathcal{V}_\zeta^{eff}(\Phi)=\kappa_3\frac{(m_\zeta^{eff})^2}{4v^2}\text{Tr}\left[\left<\Phi^\dagger\Phi\right>^2\right].
\end{equation}
in which the magnetic response has been incorporated through the medium-dependent mass, being
Eq.~\eqref{eq:magneticmass} at the lowest order, and the cumbersome
rotated electromagnetic field and the covariant derivative are no longer needed.
Such kind of effective potential is justified in the dynamics of Higgs excitation.
%%%%%%%%%%%%%%%%%%%%%%%%%%%%
%%%%%%%%%%%%%%%
The effective potential can be further simplified by plugging in Eq.~(\ref{eq:diquarkmatrix0}), i.e.,
\begin{equation}
\label{eq:deltapotential1}
\mathcal{V}_\delta \sim  \alpha' \delta^2 + \beta'\delta^4,
\end{equation}
%%\mathcal{V}_\delta \sim \alpha' \delta^2 + \beta'\delta^4 + \cdots,%
with
\begin{eqnarray}
  \label{eq:coefficients}
\alpha' \equiv (m_\zeta^{eff})^2,
\end{eqnarray}
and
\begin{eqnarray}
  \label{eq:coefficients2}
\beta' \equiv \frac{(m_\zeta^{eff})^2}{6v^2} + \frac{4 m_\phi^2}{3v^2}.
%%%\beta' \equiv (\frac{(m_\zeta^{eff})^2}{6v^2} + \frac{4 m_\phi^2}{3v^2})\kappa_3.
\end{eqnarray}
Notice that, besides the quadratic-order term of $\delta$, the quartic-order term is partially magnetic-field dependent.
As defined in Eq.~(\ref{eq:coefficients2}), it originates from the response of charged Higgs modes.
This contribution might come out of the terms less relevant than Eq.~\eqref{eq:derive},
%, say the $\Phi^4$ term
such as $\text{Tr}[(\vec{\mathcal{D}}\Phi_L)^\dagger(\vec{\mathcal{D}}\Phi_R) \Phi_R^\dagger\Phi_L]$.
%%in which both the chiral symmetry and the gauge invariance are respected.
%%


%%%%%%%%%%%%%%%%%%%
%%%%%%
As the potential of a scalar field $\delta$, Eq.~(\ref{eq:deltapotential1}) opens the possibility that
$\delta$ becomes condensed. According to the standard GL analysis, it occurs as long as the $\delta^2$
coefficient $\alpha'$ is negative and the $\delta^4$ coefficient $\beta'$ is positive.
%%%%%
Due to the magnetic effect Eq.(\ref{eq:magneticmass}), the coefficient $\alpha'$ does have chance to
take a negative value.
%%%Once $\alpha'$ becomes negative, it is not the known squared mass for Higgs octet in the CFL phase.
In this case, charged Higgs modes possess the negative energy such as $E^2(k=0)=(m_\zeta^{eff})^2<0$
and thus they behave as unstable modes.
%%%
As well known from a Mexican-hat-shape potential, unstable modes are present around the original minimum
and their excitation leads to the emergences of new condensation as well as new vacuum.
For our concerned potential Eq.~(\ref{eq:deltapotential1}), it is just the dynamical mechanism for
condensation of charged Higgs modes. By minimizing Eq.~(\ref{eq:deltapotential1}), the new vacuum
$v_\delta$ is given by
\begin{equation}
\label{eq:vacuumexpectation}
v_\delta^2 = \frac{- \alpha'} {2\beta'},
\end{equation}
and then the potential is rewritten as a Mexican-hat-shape form
\begin{equation}
\label{eq:deltapotential}
\mathcal{V}_\delta \sim - \alpha'(\delta^2 - v_\delta^2)^2.
\end{equation}
The above equations make sense for the negative $\alpha'$ mainly.
In some sense, this mechanism is analogous to the case for color-flavor-locked condensate $d$.
The latter emerges provided $\alpha$ is negative in the potential $\mathcal{V}_d \sim - \alpha (d^2 - v^2)^2$.
%%%%%%%%%%%%%%%%%
On the other hand, however, we emphasize that $\delta$ originates from the non-diagonal
elements of $\Phi$ and belongs to the color-flavor-unlocked species. This is essentially
different from the condensate $d$.
%%%%%%%%%%%%
%%%%%%%%%%%%%%%%%%%%

%%%%%%%%%%%%%%%%%%%%
As have elaborated, the sign of $\alpha'$ determines whether or not Higgs condensate emerge.
Obviously, $\alpha' \rightarrow 0$ can yield the threshold value $B_0$ for the emergence of Higgs
condensate.
Within the GL framework, there exist some of uncertainties in the numerical estimate of $B_0$.
Throughout the current work, we consider the simplifications $\kappa_3=1$ and $\beta_1=\beta_2=\beta$.
By taking $v_\perp^2=1/3$ into account and adopting the $\mathscr{O}(1)$ coefficient $\beta=\beta_2\simeq${0.5} \cite{balachandran2006semisuperfluid}, the threshold magnetic field $eB_0 = 12\beta_2 v^2 \simeq 6v^2$ is
obtained from Eq.(\ref{eq:magneticmass}).
Instead of the GL coefficient $\alpha$, further, the physical
quantity $v$ is taken as the other parameter.
If choosing $v = 50$ MeV, the threshold value of magnetic field could reach the
order of $10^{17}$~G. Numerically, such a magnitude is comparable to the estimated value of
magnetic fields in the core of neutron stars, but less than the theoretical upper
limit $10^{20}$~G inside self-bound magnetars~\cite{dong2001,lai1991cold}.
%%%%%%%%%%%%%%%

Also, the constraint $\beta'> 0$ needs to be investigated. Noticing that this coefficient is
partially magnetic-field dependent, we find that the positive $\beta'$ happens for $B < 6 B_0$ approximately.
In this sense, Higgs condensate seems to exist for the region with $B_0< B < 6 B_0$.
%%%%%%%%%%%%
If neglecting the magnetic dependence of $v$, on the other hand, $v_\delta$ is found to close with the known
CFL vacuum as the field exceeds about $3 B_0$. Keep in mind that our concerned excitations behave as the
perturbations around CFL, Higgs condensate is actually valid for relatively weak fields, say, $B < 3 B_0$.
Of course such rough estimate should not be taken too seriously.
In present scheme $v$ is treated as a parameter. This is in contrast to the
observation in phenomenological NJL models, where the change of CFL gap were studied by magnetic response of charged quarks. 
Unfortunately, the quark dof are not explicitly incorporated in the current GL framework.
We will further discuss on this point in Sec.~\ref{sec:4}.


\section{\bf Vortex solutions of Higgs condensate}
\label{sec:3}
\vspace{0.2cm}

The above-defined condensate is different from color-flavor
locked condensate in CFL vacuum.
In the view of properties of quark pairing, it is a color-flavor unlocked condensate
and has non-zero value of rotated charges.
From aspects of quasiparticle excitation, it is
defined as a fluctuation around CFL vacuum.
More specially, it is charged Higgs field originated from $SU(3)_{C+F}$
 symmetry spontaneously breaking.
If the magnetic field is included,
those charged Higgs field is unavoidably affected by the magnetic field.
We  have already token a crude estimate to determine whether or not the 
charged Higgs field condenses.
As discussed in Sec.~\ref{sec:2}, the energy cost of creating the Higgs-like condensate
is mainly determined by Eq.~\eqref{eq:deltapotential1}.
From Eq.~\eqref{eq:deltapotential1}, 
it is found that the charged Higgs field is excited 
to minimize the system energy
and condensed when $B >B_0$.


When the magnetic field is getting larger, the original
symmetry pattern is broken further, $SU(3)_C \times SU(3)_F \rightarrow SU(2)_{C+F}$.
For the homogenous state, it is called as the MCFL
phase~\cite{ferrer2005magnetic,ferrer2006magnetic}. 
Owing to the original symmetry is broken,
the CFL vacuum  disappear and then the charged  Higgs-like condensate 
built on the CFL vacuum  is difficult to be realized.



As for possible inhomogeneous solutions of the
Higgs-like condensate,
from the discussion of Sec.~\ref{sec:2}, we can notice that the
transverse momentum space distribution has been modified by 
the magnetic field.
It may lead to  inhomogeneous condensate solutions 
have such $\sim e^{-ik_y y}f(x)$ form in our convention.
From experience of the conventional type II superconductivity,
these topological solutions
prefer periodic lattice domains (e.g. Abrikosov vortices) to minimize the energy.
Owing the same reason that 
the original symmetry is broken, such Abrikosov-like inhomogeneous solutions
is also invalid.




  



Although the original symmetry pattern  is broken,
It does not mean that these Higgs-like inhomogeneous solutions are not possible.
In fact, there exist possible mechanism to  make
 topological objects realize in the core of CFL vortices. 
For the CFL vortices, they are originated from symmetry spontaneously breaking, 
for example  $U(1)_B$ superfluid vortices and the non-Abelian CFL vortices
originated from $U(1)_B$ 
and $SU(3)_{C+F}$ symmetry spontaneously breaking respectively.
One of important properties of these CFL vortices is that
the emergence of those CFL vortices is regardless with an external magnetic field.
Therefore it is convenient for us to choose
those CFL vortices solutions as background in the following discussions.
In a given CFL vortices background, when the color-flavor locked symmetry is violated,
it was found that
the diagonal elements in the CFL vortices solutions are not equivalent, 
for example in the core of
non-Abelian vortices~\cite{balachandran2006semisuperfluid,nakano2008non,eto2009color}.
In~\cite{iida2005magnetic}, it  pointed out similar situation that the color-flavor
unlocked species  emerge in the core of a $U(1)_B$ vortices in the presence of the magnetic field
\footnote{Different from our cases, they have introduced external electromagnetic field and
gluon field rather than only the rotated electromagnetic field}.
As for us, although the original symmetry pattern is broken, the residual symmetries $SU(2)_{C+F}$
allow the presence of a color-flavor unlocked condensate in the core of
CFL vortices.
As a consequence,
it is important for us to check the possibility of Higgs-like condensate and the corresponding
topological structure in the core of the given CFL vortices background.


% The above-defined condensate is difficult to be realized in the homogenous case. Homogenous Higgs condensate
% is usually unstable since it corresponds to a violation of the $SU (3)_{C+L+R}$ symmetry.
% %%%%%%%
% %%%%%%%%%%%%%%%%%%
% Even so, the realization of Higgs condensate as inhomogeneous vortices is possible.
% Note that the existence of topological objects is just the consequence of spontaneous
% symmetry breaking. As long as the $\Phi$ configuration in color-flavor-locked-type matter is assumed properly,
% we can consider the spontaneous breaking to Higgs condensate and then derive the vortex solution
% made up of $\delta$ from the present GL Lagrangian.

Before going into the detail, we first give a brief review on the CFL vortices.
As seen in Eq.~(\ref{cfl}), the original QCD symmetry $G$ is broken to
the color-flavor locked symmetry $H$.
%%\begin{equation}
%  \label{eq:hgroup}
% H =SU(3)_{C+F} \times Z_3.
%\end{equation}
Thus, the diquark matrix $\Phi$ can be parameterized in the
topological space
\begin{equation}
  \label{eq:cflvortexgroup}
  \frac{G}{H} \simeq \frac{SU(3) \times U(1)_B}{Z_3}  \simeq U(3),
\end{equation}
where $Z_3$ is a discrete symmetry.
Since the symmetry $U(1)_B$ is broken spontaneously, on the one hand, a superfluid vortex can be generated.
In the cylindrical coordinates, the spatial configuration for a minimal-wound CFL vortex reads
\begin{equation}
  \label{eq:bvortexphi}
\Phi =vf(r)e^{i \theta} \texttt{diag}(1,1,1),
\end{equation}
and equivalently
\begin{equation}
 d = vf(r)e^{i\theta}. \label{eq:bvortex}\end{equation}
%due to Eq.~(\ref{eq:phi}).
There, the polar angle $\theta$ originates from $U(1)_B$ breaking and
$f(r)$ is the profile function with the boundary
conditions $f(0) = 0$ and $f(\infty) =1$.

Because of the non-Abelian property of Eq.~(\ref{eq:cflvortexgroup}), on the other hand,
there should exist more complicated structure for the diquark-condensate matrix $\Phi$.
For instance, the diagonal matrix elements might not be degenerate and/or the non-diagonal
elements might make sense under some specific circumstances.
In recent years, the topological object called non-Abelian vortices has been suggested for
CFL. Its typical minimal-wound solution is given
by~\cite{balachandran2006semisuperfluid,nakano2008non,eto2009color}:
%%%%%%%%%%%
%%%%%%%
\begin{equation}
  \label{eq:nvortex}
  \Phi = v\begin{pmatrix}
   f(r)e^{i\theta} & & \\ & g(r) & \\ & & g(r)
  \end{pmatrix}.
\end{equation}
%%which may be described by two of independent vortex solutions also~\cite{balachandran2006semisuperfluid}.
The characteristic of the non-Abelian vortices is the color-flavor-locked symmetry breaking.
In the vicinity of core of such kind of vortices, it is pointed out that the locked symmetry $H={SU(3)_{C+F}}$ is broken to
$H' =U(1)_{C+F} \times SU(2)_{C+F}$ \cite{nakano2008non,vinci2012spontaneous}.
%%%namely $H={SU(3)_{C+F}} \rightarrow H' =U(1)_{C+F} \times SU(2)_{C+F}$
%%
The collective modes associated with this breaking are of the orientational zero modes and they appear in the
topological space
\begin{equation}
  \label{eq:cp2}
 \frac{H}{H'} = \frac{SU(3)_{C+F}}{U(1)_{C+F} \times SU(2)_{C+F}} = CP^2.
\end{equation}
%%%%%%%%%%%%%%%%%%%%%%%%%

The two kinds of CFL vortex solutions could be generated spontaneously regardless of an
external magnetic field.
When the applied fields such as $B > B_0$ are introduced, let us assume that Higgs condensate
emerges in the vicinity of the core of non-Abelian vortices. Note that the color-flavor-locked
symmetry might be broken as \cite{ferrer2007magnetic}
\begin{eqnarray}
H\rightarrow M=SU(2)_{C+F},
\label{cfl2}\end{eqnarray}
in the presence of magnetic fields.
If enforcing such magnetic-induced breaking, the remaining subgroup $H'$ is broken to $M =SU(2)_{C+F}$.
In this case, the Higgs condensate inside the CFL vortex core is parameterized in the topological space
\begin{equation}
  \label{eq:mcfsymm}
 \frac{H'}{M} = \frac{SU(2)_{C+F} \times U(1)_{C+F}}{SU(2)_{C+F}} = U(1)_{C+F}.
\end{equation}
It underlines that the $U(1)_{C+F}$ breaking is possibly responsible for
the Higgs-condensate emergence. We will be concerned with such a specific ansatz in the rest of the paper.
%%%%%%
%To this end, we emphasize that the explicit symmetry breaking $H \rightarrow M$ itself does not lead to %
%%%%%%%%%%%%%%%%%%%%%

\vspace{0.2cm}
\textbf{A. Superfluid vortex formation }
\vspace{0.2cm}

In order to explore the vortices made up of a condensed $\delta$ field, it is convenient to consider the effective Lagrangian
\begin{equation}
\label{eq:mcflvorticehamilton}
 \mathcal{L}_\delta = (\partial \delta)^* (\partial \delta) - \frac{\alpha'}{4}\delta^2 - \frac{\beta'}{4}\delta^4,
\end{equation}
and equivalently
\begin{equation}
  \label{eq:vortond}
  \mathcal{L}_\delta = (\partial \delta)^* (\partial \delta) + \frac{\alpha'}{8 v_\delta^2}(\delta^2 - v_\delta^2)^2.
\end{equation}
As stressed in Sec.~\ref{sec:2}, the influence of rotated magnetic fields has been attributed to the in-medium
coefficients ${\alpha'}$ and ${\beta'}$ so that the usual derivative term is adopted in above equations.
The simplest possibility is that, just like the $U(1)_B$ vortex for CFL, a superfluid vortex solution can be
generated from $U(1)$ breaking. Similar as Eq.~(\ref{eq:bvortex}), the vortex (string) configuration reads
\begin{equation}
	\delta= v_\delta f(r) e^{i\theta},\label{mcflvortex}
\end{equation}
where the phase angle arises from $U(1)_{C+F}$ breaking.
%%%%%%%%%%%%%%%%%%%%

Let us first discuss qualitatively the formation of such a string. According to our assumption,
Higgs condensate exists in the core region of non-Abelian CFL vortices. It implies that the
vortex-core size of non-Abelian vortices should be larger than that of $\delta$ string. As a typical
$U(1)$ vortex, the latter might be described by the characteristic radius $R_\delta$, namely the
correlation length of Higgs condensate.
Also it is noticed that, in the scope of CFL vortices, the radius $R_d$ for $U(1)_B$ vortices generally
exceeds the mean radius of non-Abelian vortices. In this sense, a simple comparison between the two
characteristic radii, say $R_d > R_\delta$, is required for the $\delta$-string generation.
%%%
Moreover, it is necessary to further examine the formation condition and the magnetic-field dependence
for $\delta$ string. It is well known that, for a $U(1)$ string, the magnitude of $R$
%%reflects correlation length of the condensate and its
could be estimated by an inverse mass of the concerned Higgs mode~\cite{vilenkin2000cosmic}.
%associated with $U(1)$ breaking
%%%%%%%%%%%
In the CFL case, $R_d$ is determined by the inverse mass of $\phi$ mode defined above and it can be
found that $R_d \simeq (-\alpha/3)^{-1/2}$.
%%%%%%
%is decided by $m_\phi^{-1}$  (since the Higgs mode $\phi$ is responsible for $U(1)_{B}$ breaking).
%From the GL Lagrangian and/or the effective Lagrangian of $d$ (see the Eq.~(\ref{eq:vortonb}) below), the magnitude of $R_d$ %can be given to be $(-\alpha/3)^{-1/2}$ and it is
%%%%%%
As for the case of $\delta$ string, the estimate $R_\delta \simeq (-\alpha'/4)^{-1/2}$ can be obtained from
Eq.~(\ref{eq:mcflvorticehamilton}), although the Higgs mode associated with $U(1)$ breaking is not $\zeta$
mode.
%%%%%%%%%%
%Note that in our case the $\zeta$ mode is defined in CFL and it is not associated with $U(1)_{C+F}$
%breaking.%%square mass of $\zeta$ octet (being negative in the presence of magnetic field) does not decide the radius
%%$R_\delta$. %%%%%%%%%%%%%%%%%%%%%%%%%%%%%%%%%%%
%%%%%%%%%%%%%%%%%%%%%%%%%%%%%
Therefore, the requirement $R_d > R_\delta$ can be turned into the relation $-3\alpha' > -4\alpha$,
which is the formation condition for $\delta$ string. As the consequence, the magnetic fields fulfilling
the condition is about $B > \frac{7}{3} B_0$ with $\beta_1=\beta_2=\beta$ and $\kappa_3=1$.
%%%%%$eB > - 4\alpha+12\beta_2 v^2$. Numerically, the region is about
%%%%%%%%%%

By inserting Eq.(\ref{mcflvortex}) into the Euler-Lagrange equation, we then obtain the profile
function of $\delta$ string
\begin{equation}
\label{eq:profilefunction}
 f'' + \frac{f'}{r} -\frac{f}{r^2} - (\frac{\alpha'}{4} + \frac{\beta'}{2} v_\delta^2 f^2)f=0,
\end{equation}
where $f'$ and $f''$ denote the first- and the second-order derivatives of $f(r)$ with respect to $r$,
respectively. Similarly, we apply the solution Eq.~(\ref{eq:bvortex}) to derive the $d$-string profile.
%%%%%%%%%%%%%%consider a simple $U(1)_B$ vortex solution only
%and yield the profile equation
%\begin{equation}
%  \label{eq:bvortexprofile}
%  f'' + \frac{f'}{r} -\frac{f}{r^2} - (\frac{\alpha}{3} + \frac{8}{3}\beta v^2 f^2)f=0.
%\end{equation}
%in a similar way.
% the Lagrangian of $d$.
%%%%%for $d$-string
%%\begin{equation}   \label{eq:bvortexlag}
% \mathcal{L}_d = (\partial d)^* (\partial d) - \frac{\alpha}{3} d^2 - \frac{2 \beta}{3} d^4.
%\end{equation}
%%%%%%%%%%%%%%%%%%%%
%Eq.~(\ref{eq:profilefunction}) is magnetic-field dependent as the coefficients $\alpha'$ and $\beta'$ are involved.
%% the properties for  are expected different.
In Fig.~\ref{fig:2}, the results of $\delta$ string with different magnetic fields and the
$d$-string result (being magnetic-field independent) are plotted. 
%%%Note that the significant difference between two kinds of profiles.%
As shown in Fig.~\ref{fig:2}, it is clear that $R_d$ is far larger than $R_\delta$ since
the magnetic fields in the latter are chosen typically large. At the same time, it is
observed that $R_\delta$ tends to be suppressed for larger fields.
%%%%%%%%%%%%%%%%%%%
The change in spatial size of $\delta$ string is easily explained from the medium coefficient $\alpha'$.
%%%%% 

Finally, we briefly discuss the kinetic energy of $\delta$ string per length unit.
%%Physically, such kind of linear tension might be roughly given by the area of profile function versus $r$. From Fig.~\ref{fig:2},
%%the difference in areas under two $\delta$-profiles indicates the existence of the $B$ dependence of string tension.
%%%%%%%%%%%%%%%%%%
%%Without loss of generality, the definition of string tension is expressed as
%\begin{equation} \mathcal{T} = \int^{2\pi}_{0}d\theta \int^L_{R_\delta} \mathcal{H} rdr \label{eq:tension},
%\end{equation}
%where $\mathcal{H}$ denotes the system Hamiltonian.the $\delta$-profile equation
By using Eq.(\ref{eq:profilefunction}), we ignore the constant contribution and give the
asymptotic expression of string tension
\begin{equation}
  \label{eq:tension1}
  \mathcal{T} \sim v_\delta^2 ln\frac{L}{R_\delta}.
\end{equation}
%%%%which can also be derived from the quantization of $U(1)$ string.
As a cut-off constant, $L$ is introduced to account for the total radius of a superfluid vortex.
%%%%%
Note that $v_\delta$ and $R_\delta$ have the opposite medium dependence, the tension energy
behaves as an increasing function of $B$.
Also, the logarithmic-divergent tendency becomes much obvious for stronger fields.
%%%
% so that such kind of straight, global string might have instability in a
%% made up of the Higgs condensate only.
%%%%%%%%%%%%%%%%%%%

\begin{figure}
	\includegraphics[width=4in]{2.eps}
	\caption{The profile functions of $\delta$ string with $B = 4B_0$ (red solid line) and
    $B = 5 B_0$ (green dashed line) and the profile function of $d$ string (dotted line).}
	\label{fig:2}
\end{figure}
\vspace{0.2cm}
\textbf{B. Formation of vorton structure }
\vspace{0.2cm}

Now we turn to a more complicated situation where both of condensates ($\delta$ and $d$ ) exhibit the
spatial-dependent properties simultaneously.
There exists the theoretical possibility that the two strings from $U(1)$ breaking allow for the existence of
a topologically and energetically stable vorton. The scenario of vortons ( known as string loops, vortex rings also )
was first considered in the scope of cosmic string~\cite{vilenkin2000cosmic,witten1985superconducting,davis1988physics1,davis1988physics2,haws1988superconducting}.
For color superconducting quark matter, it was studied in the physical environment with quark flavor asymmetry~\cite{kaplan2002charged,buckley2002superconducting}. There the condensations of two Nambu-Goldstone
modes, say $K^0$ and $K^+$ condensates, were introduced. As the $K^+$ and $K^0$ condensed strings are generated
from $U(1)$ breaking, their coexistence is possible to support a stable vorton.
%%%%%%%%%%
%the vorton formation  case of introducing the condensates of , in the CFL environment. 
%electromagnetic $U(1)$ and hypercharge $U(1)$ symmetry breaking, respectively. In Refs.\cite{kaplan2002charged,bedaque2011vortons,buckley2002superconducting}, \cite{bedaque2011vortons}.
%Even though these condensates are not mentioned,
%%%%%%%%%%%%%%%
Now that two condensates ($d$ and $\delta$) are under consideration,
the essential physics behind should share some analogy with that discussed in literature.
%%%%

Our starting point is that the $d$ vortex is generated from $U(1)$ breaking while the $\delta$ condensate emerges
inside the vortex core. At this stage, the former is regarded as a usual, straight string parallel to the
$z$ direction. Similar to the Eq.~(\ref{eq:bvortex}), it has the form of $d = d(r)e^{i\theta}$.
Along the $z$ direction, the $\delta$ condensed field might carry non-vanishing current and charge.
Without losing generality, the form of such a nontrivial $\delta$ vortex solution reads
\begin{equation}
  \label{eq:delta}
  \delta =  e^{i(kz+\omega t)}\delta(r).
\end{equation}
In the minimal-wound case the wave number $k$ contributes to the current $J$ via
$J =k\int dz \int dS \delta^2$, where $S$ denotes the area being perpendicular to $z$ axis.
Similarly, the frequency $\omega$ might contribute to the conserved
Noether charge $Q$ via $Q = \omega\int dz \int dS \delta^2$.
%%% \begin{equation}
%  \label{eq:vortonquantumq}
%Q = \omega\int dz \int dS \delta^2,
% \end{equation}.
%%%%%%%%%%%%%%%%%

For the purpose of exploring a possible vorton, we will consider the simplest case
with two order parameters, say a $U(1) \times U(1)$ model Lagrangian
$\mathcal{L}(d,\delta)= \mathcal{L}_d +\mathcal{L}_\delta + \mathcal{L}_{d\delta}$.
%%%%%%%%%%%%%%
The Lagrangian of $d$ field is easily written in the Mexican-hat form
\begin{equation}
  \label{eq:vortonb}
  \mathcal{L}_d  = (\partial d)^* (\partial d) +\frac{\alpha}{6 v^2}(d^2 - v^2)^2,
\end{equation}
where the known vacuum Eq.~(\ref{eq:dvaccum}) has been used to eliminate the coefficient $\beta$.
For the $\delta$ field its Lagrangian has been given by Eq.~(\ref{eq:vortond}).
%%%%%%%%%
The term with $d$ and $\delta$ mixing, being important for a $U(1) \times U(1)$ model,  may be
formally written as
\begin{equation}
  \label{eq:vortoninter}
  \mathcal{L}_{d\delta} = -\lambda d^2 \delta^2,
\end{equation}
where $\lambda$ is required positive.
%%
Recall the original GL Lagrangian with the matrix $\Phi$, this term needs to be associated with the contributions from $\alpha$ and $\beta_2$ (exactly, $\alpha'$)
at the same time. In order to guarantee that the resulting Lagrangian is theoretically
under control and resembles a $U(1) \times U(1)$ model, we pick up the relevant terms
in expansion of Eq.~(\ref{gl}) and define the coefficient as
\begin{equation}
  \label{eq:vortoninter1}
  \lambda = -\frac{1}{3 v^2}(\alpha' +\frac{\alpha}{3}).
\end{equation}

The first task is to investigate the conditions for existence of $\delta$ condensate at the $d$-string centre.
Apparently, Eq.~(\ref{eq:delta}) leads to the changes in the formalism.
For instance, the effective potential for $\delta$ field becomes
\begin{equation}    \label{eq:vortond1}
  \mathcal{V}_\delta= -\frac{\alpha'}{8v_\delta^2} [\delta^2 - (v_\delta^2 - \frac{8v_\delta^2}{\alpha'}(\omega^2 -k^2))]^2 \ .\end{equation}
%%%%%%%%%%%
In the vacuum with $d \neq 0$ and $\delta = 0$, the effective potential provides the additional
contribution to the Lagrangian $\mathcal{L}_d$. To guarantee the $d$-field symmetry is broken such
that $d \neq 0$, it is necessary to require that the vacuum contribution is positive.
By considering the constant terms in Eqs.~(\ref{eq:vortonb}) and (\ref{eq:vortond1}), the condition reads
\begin{equation}
\label{eq:vortonb2}
  \frac{\alpha}{6} v^2 < \frac{\alpha'}{8v_\delta^2}(v_\delta^2  - \frac{8v_\delta^2}{\alpha'}(\omega^2 -k^2))^2.
\end{equation}
In such a vacuum, also, we requires that the quadratic
coefficient of $\delta$ is positive to guarantee the
$\delta$-field symmetry remains unbroken.
By considering the relevant term in Eq.~(\ref{eq:vortond1}) and taking the mixed
term into account, the other condition is yielded,
\begin{equation}
  \label{eq:vortonb3}
  \lambda v^2 + \frac{\alpha'}{4}- \omega^2 +k^2 > 0.
\end{equation}
%%%%%%%%%%%%%%%%%%%%%
On the other hand, the above conditions are not sufficient to yield the vacuum
with $d = 0$ and $\delta \neq 0$.
%% which emerges at the centre of $d$ string. 
At classical level the requirement that the $\delta^2$ coefficient is negative results in
\begin{equation}
  \label{eq:vortonb4}
\omega^2 - k^2 -\frac{\alpha'}{4} > 0 \ ,
\end{equation}
where the mixed term with $\lambda$ does not take effect. When regarding $\delta$ as
the quantum solution and taking its gradient energy cost into account, one requires that the
ground state possesses a negative eigenvalue (see, e.g.,~\cite{vilenkin2000cosmic,haws1988superconducting}
for details).
Instead of Eq.~(\ref{eq:vortonb4}), a more accurate form of the sufficient condition should be given by
\begin{equation}
  \label{eq:vortonb5}
  \omega^2 - k^2 -\frac{\alpha'}{4} > \sqrt{- \frac{2}{3}\alpha \lambda v^2}.
\end{equation}
%%%%%%%%%%%%%%%%%%%%
%%%%%%%%%
%%%%%%%%%%%
Only if the conditions Eqs.~(\ref{eq:vortonb2}), (\ref{eq:vortonb3}) and (\ref{eq:vortonb5}) are
satisfied at the same time, there exist not merely the $d$ condensate but also the $\delta$ condensate.

Then, we investigate the profile properties with the boundary conditions $\delta(r \rightarrow 0) = v_\delta$ and $\delta(r \rightarrow \infty) = 0$ as well as the usual boundaries for $d$.
Based on the Lagrangian $\mathcal{L}(d,\delta)$, it is easy to obtain the profile functions of $d$ and $\delta$ condensates from
\begin{equation}
  \label{eq:deuler}
  d'' +\frac{d'}{r} - \frac{d}{r^2} - (\lambda \delta^2 + \frac{\alpha}{3})d + \frac{\alpha}{3v^2}d^3 = 0,
\end{equation}
and
\begin{equation}
  \label{eq:beuler}
  \delta'' +\frac{\delta'}{r} - \frac{\delta}{r^2} - (k^2 - \omega^2)\delta - (\lambda d^2 + \frac{\alpha'}{4})\delta + \frac{\alpha'}{4v_\delta^2}\delta^3 = 0 \ ,
\end{equation}
respectively. In principle, not only the mixed term with $\lambda$ but also the contribution from $k^2-\omega^2$ play their roles as seen in Eq.~(\ref{eq:beuler}). For certainty, we only consider the limit of $k^2=\omega^2$ which corresponds to a
critical situation for our concerned vortex solution (the so-called ``chiral" case~\cite{lemperiere2003behaviour}).
In this limit, three of conditions imposed on the magnetic-dependent coefficients $\alpha'$ and $\lambda$ are very stringent.
Indeed, we find that the magnetic-field values fulfilling the conditions is actually confined to a
rather narrow region.
It means that there is no need to discuss the magnetic-field dependence of the profile functions in the remainder of the present work.
%%%%%%%%%%%%%%%%%%%%%%%%%%%%%%%%%%
%%%%%%%%%%%%%%%%%%%%%%%%%%%%%
With help of the parameters used in Sec.~\ref{sec:2}, the appropriate field is found to be about $eB \simeq 2 eB_0$ numerically. This is a weak magnetic background at which $v_\delta$ is much smaller than $v$.
%%%%
With the boundary conditions two profile functions are plotted in Fig.\ref{fig:3}.
As expected the non-vanishing $\delta$ field exists inside the core region of $d$ string.
It is observed that the radius $L$ for $d$ string is larger than the radius $L_\delta$ for the
$\delta$ vortex solution.
%%%%%%
The similar behavior as Fig.\ref{fig:3} had been obtained in the literature, which opens the
possibility of a nontrivial vorton structure.
%%%%%%%%%%%%%%%%%%%%%%%%%%%%%%%%%%%%

\begin{figure}
	\includegraphics[width=4in]{3.eps}
	\caption{Dimensionless profile functions $f(r)=d(r)/v$ (blue dotted line) and
     $f(r)=\delta(r)/v_\delta$ (red solid line) in the $\omega^2 = k^2$ limit.
     }
	\label{fig:3}
\end{figure}

Third, it is time to introduce a proper spatial configuration and then
construct a stable vorton state.
To obtain a finite energy for the $d$ vortex, the simple choices are that it exists in a finite
container or it forms a closed circular loop.
Our concerned configuration is the latter case. When a straight $d$ string is bent to a
closed loop (ring), $z$ denotes the direction along a ring and the ring radius becomes $L$.
Consequently, the phase change of $\delta$ field and thus the charge/current happen in
the ring direction.
%%%%%%%%%%%%%%%%
Now that the linkage of two vortices has been realized, we focus on energy of the resulting vorton state.
Let us first discuss the contributions from $d$ and
$\delta$ respectively. For the $d$-string, its length is $2 \pi L$ as a closed loop.
Suppose its linear tension is $\mathcal{T}_d$, the energy for $d$ vortex is given by
$\mathcal{E}_d = 2\pi L \mathcal{T}_d$. As seen, this energy is mainly decided by $L$.
%%%%%%
If the system were made up of $d$ singly, it would prefer to shrink
rather than expand.
%%%%
On the other hand, the $\delta$-vortex energy can be generally expressed as
\begin{equation}
  \label{eq:deltah}
  \mathcal{E}_\delta = \int dz \int dS  (\nabla_r \delta)^2 + (k^2 + \omega^2)\delta^2 + (\lambda d^2 + \frac{\alpha'}{4})\delta^2 - \frac{\alpha'}{8v_\delta^2}\delta^4.
\end{equation}
%%%%%%%%%%%%%%
By considering the profile equation Eq.(\ref{eq:beuler}), it is reduced to
\begin{equation}
  \label{eq:energydelta}
  \mathcal{E}_{\delta} =  2\pi L\frac{\alpha' \Sigma_4}{8v_\delta^2} + 4 \pi L \omega^2 \Sigma_2,
\end{equation}
where $\Sigma_2$ and $\Sigma_4$ are short for
$\Sigma_2 = \int dS \delta^2$ and $\Sigma_4 = \int dS \delta^4$ respectively.
In view of the fact that the quantity $Q$ is conserved during variation of $L$, the energy can be further simplified as
\begin{equation}
  \label{eq:energydelta1}
  \mathcal{E}_{\delta} =  2\pi L\frac{\alpha' \Sigma_4}{8v_\delta^2} + \frac{Q^2}{\pi L \Sigma_2}.
\end{equation}
Note that the last term in RHS of Eq.(\ref{eq:energydelta1}) has the
$L^{-1}$ behavior.
This implies that the system made up of $\delta$ might prefer expansion of a vorton.


As the result, two competitive tendencies, shrinking and expanding, 
make an energetically-stable vorton state possible. We minimize the total energy 
$\mathcal{E} = \mathcal{E}_\delta + \mathcal{E}_d$ with respect to $L$, and the stabilized radius 
$L_0$ is hence:
\begin{equation}
\label{eq:vortonr}
 L_0^2 = \frac{Q^2}{2\pi^2\Sigma_2(\mathcal{T}_d +
   \frac{\alpha' \Sigma_4}{8v_\delta^2})}.
\end{equation}
At the radius $L_0$, the stable energy for a vorton state is a function of $Q$ and $L$, say, 
$\mathcal{E}_0 = 2 Q^2/(\pi L_0 \Sigma_2)$.
%%
The present vorton state is, in a sense, an extension of the straight $\delta$ string discussed 
in Sec.~\ref{sec:3}A. The magnetic field in the former is weak, allowing the charged, 
color-flavor-unlocked fluctuations to be excited around the CFL vacuum.
%%
Moreover, the vorton energy is no longer divergent and the object proposed here might become 
energetically stable under some astrophysical circumstances. Therefore, the current scenario of vortons 
is a more plausible realization of Higgs condensate compared with a simple string case.
%%%%%%%%%%%%%%%%%%%%%%%%%%%%%%%%%%%%%%%%%%%%%%%%%%%%%


%%%%%%%%%%%%%%%%%%%
\vspace{0.2cm}
\section{\bf Conclusions and discussions}
\label{sec:4}
\vspace{0.2cm}

Having explored the magnetic effect of charged Higgs modes and the possible formation of vortices, 
we briefly discuss several implications of the results and open questions in future research.

\emph{1. Magnetic effects in GL and NJL.}
We have predicted the emergence of Higgs condensate within the GL framework.
Besides the diagonal matrix elements, we are concerned with the charged, non-diagonal elements in the GL 
order parameter.
As is discussed in Sec.~\ref{sec:2}, the latter become excited through magnetic influence on
the traceless potential, in particular magnetic response of the charged Higgs modes.
The resulting Higgs condensate originates from the possible violation of $\Phi_L=\Phi_R$ and is of the 
color-flavor unlocked species.
Different from color-flavor-locked species being the spin-$0$ condensate, 
the unlocked species appear to exhibit a few features suggestive of the charged vector boson condensate.
Even though its detailed microscopic content is still unclear, this point might be helpful for explaining why
the squared mass of charged modes decrease in the presence of a magnetic field.
%%%%%%%%%%%
%%%%%%%%%%%%%%%%%%%%%%%%%%%%%%%% 

Within the phenomenological NJL models, on the other hand, the neutral, color-flavor-locked species 
were concerned in literature.
Due to the magnetic response of charged quarks, the VEVs of color-flavor-locked diquark condensate need 
to be differentiated. It leads to the gap splitting and a less-symmetric \emph{magnetic color-flavor-locked phase} 
(MCFL).
%%%%%%%%%%%%%
From the NJL model side, the de Haas-van Alphen oscillations of gaps were illustrated~\cite{ferrer2005magnetic,fukushima2008color}.
%although some recent studies suggest that parts of unphysical oscillations may be eliminated by the appropriate regularization scheme ~\cite{allen2015magnetized}.
The gap equations were derived analytically in the limit of strong magnetic 
field~\cite{ferrer2006color,sen2015anisotropic}. There, the ``strong field" means that $eB$ has the order 
of the square of quark chemical potential at which only the lowest Landau level of quarks is occupied.
%%$\mu_q^2$%%
It was found that the magnetic field enhances the relevant color superconducting gap,
which might be considered as a generic magnetic catalysis.
%%%%%%

We emphasize that the magnetic effect under consideration in present
paper is totally different from the MCFL phenomena. Throughout the whole work,
we regard the most-symmetric CFL phase with a uniform, flavor-independent gap as the known solution 
and ignore the gap splitting as well as the possible oscillations.
The model-independent treatment is essentially an effective theory concentrating on corrections to the 
CFL results. It only allows for qualitative discussions of changes around the CFL vacuum.
Once the magnitude of $eB$ is so large that MCFL replaces CFL to be the ground state, the symmetry 
breaking pattern might have been broken explicitly. In this case, the GL formalism based on Eq.(\ref{cfl}) 
is no longer valid and the additional Higgs condensate does not make sense anymore.
Because of the absence of quark dof, indeed, the current GL approach can not handle the detailed magnetic effect, 
say, the magnetic-induced contribution from charged quarks.
%%%%%%%%%%%%%%%%%%%%%
So far it is an open question how to understand the implications of Higgs condensate from the viewpoint of
quark-photon interactions and to bridge ``gaps" between the two kinds of magnetic effects in a consistent manner.
%%%%%%%%%%%%%%%%%%
%%%%%%%%%%%%
%%%%%%%%%%%%%%%%

\emph{2. Vortices in color superconducting quark matter}. Assuming that Higgs condensate exists inside the core 
region of the CFL vortices, we simplify the relevant symmetry breaking as a $U(1)$ breaking. Based on the symmetry 
consideration, a superfluid-like vortex string made up of $\delta$ can be constructed and then the discussion is 
extended to the situation where not merely $\delta$ but also $d$ are incorporated.
There still exist some aspects which are not dealt with in the present study of stable vortons.
For instance, the profile behaviors have been obtained from the critical situation $\omega^2 = k^2$.
Once the effects of $\omega^2\neq k^2$ is introduced, nevertheless, existences of $Q$ and $J$ and 
detailed electromagnetic properties for vortons need to be examined.
At the same time complicated curvature effect might appear, if the spatial thickness for $d$ string 
is not very large with respective to that for $\delta$. In this case, it is necessary to build more 
realistic model for vortons' structure.
%%%%%%%%%%%%%
No doubt that more complexities is expected from the realistic situations. 
For instance, when a large strange-flavor mass is considered, the color-flavor-locked matter with $K$ 
condensates and the resulting vortex solutions need to be included as well~\cite{kaplan2002charged,buckley2002superconducting}.
%%%%%%%%%%%%%%%%%

On the other hand, the rich physics lies in the interplay of rotated gluons, which originated 
from the rotated electromagnetic mechanism, with the applied field $B$.
By considering magnetic responses of gluons (through their rotated charges), this issue was studied 
within gluon mean-field theory at high densities~\cite{ferrer2006magnetic}.
There, the vortices with gluon condensate were suggested for very strong (external) magnetic fields.
In the present paper we have ignored the rotated gluons with heavy Meissner mass within the GL framework.
Once the gluon dof is taken into account, nevertheless, the anti-screening effect predicted in 
Ref.~\cite{ferrer2006magnetic} might make the current vortices with 
Higgs condensate difficult to generate.
In addition, the gluon interactions and the gluon-photon mixing were found to be important for the 
stability and electromagnetic properties of vortices with color-flavor-locked diquark condensate~\cite{vinci2012spontaneous,eto2010instabilities,iida2005magnetic}.
So far it remains unclear whether or not there exists the internal links among the gluon-condensate vortices, 
the CFL vortices, as well as the Higgs-condensate vortices. Together with the above-mentioned topics, further 
studies in the astrophysical ``Laboratories" such like the interior of compact stars are interesting undoubtedly.
%%%%%%%%%%%%%%%%%%%%%%%%%%



\vspace{0.5cm} \noindent {\bf Acknowledgements} \vspace{0.5cm}

This work was supported by National Natural Science Foundation of
China ( NSFC ) under Contract No. 10875058.

\vspace{0.7cm}

%%%%%%%%%%%%%%%%%%%%%%%%%%%%%%%%%%%%
\begin{thebibliography}{99}

\bibitem{andersen2016phase}
Jens~O Andersen, William~R Naylor, and Anders Tranberg.
\newblock Phase diagram of QCD in a magnetic field.
\newblock {\em Reviews of Modern Physics}, 88(2):025001, 2016.

\bibitem{kharzeev2013strongly}
Dmitri~E Kharzeev, Karl Landsteiner, Andreas Schmitt, and Ho-Ung Yee.
\newblock Strongly interacting matter in magnetic fields: a guide to this
  volume.
\newblock In {\em Strongly Interacting Matter in Magnetic Fields}, pages 1--11.
  Springer, 2013.

\bibitem{miransky2015quantum}Vladimir~A Miransky and Igor~A Shovkovy.
\newblock Quantum field theory in a magnetic field: From quantum chromodynamics  to graphene and dirac semimetals.
\newblock {\em Physics Reports}, 576:1--209, 2015.

\bibitem{kharzeev2008}
D. E. Kharzeev, L. D. McLerran and H. J. Warringa.
\newblock The effects of topological charge change in heavy ion collisions: event by event P and CP-violation.
\newblock {\em Nuclear Physics A}, 803:227-253, 2008.
%Nucl. Phys. {\bf A 803}, 2008 227-253.
% [arXiv:0711.0950].

\bibitem{skokov2009}
V. Skokov, A. Y. Illarionov, V. Toneev.
\newblock Estimate of the magnetic field strength in heavy-ion collisions.
\newblock {\em Int. J. Mod. Phys. A}, 24: 5925-5932,2009.
% [arXiv:0907.1396].

\bibitem{dong2001}
D. Lai.
\newblock Matter in strong magnetic fields.
%\href{http://dx.doi.org/10.1103/RevModPhys.73.629}
\newblock {\em Reviews of Modern Physics},73(3): 629-661,2001.


\bibitem{lai1991cold}
Dong Lai and Stuart~L Shapiro.
\newblock Cold equation of state in a strong magnetic field-effects of inverse
  beta-decay.
\newblock {\em The Astrophysical Journal}, 383:745--751, 1991.

%\bibitem{bali2012}G. Bali, et al. \newblock The QCD phase diagram for external magnetic fields.
%\newblock{\em Journal of High Energy Physics}, 1202:044-048,2012.
%\newblock QCD quark condensate in external magnetic fields.
%\newblock{\em Physical Review D}, 86:071502(R), 2012.
%G. Bali, F. Bruckmann, G. Endrodi, F. Gruber, A. Sch\"{a}fer. \newblock Magnetic field-induced gluonic (inverse) catalysis and pressure (an)isotropy in QCD. \newblock {\em Journal of High Energy Physics}, 1304:130-135, 2013.
%%

%%%\bibitem{miransky2002} V. A. Miransky, I. A. Shovkovy. \newblock Magnetic catalysis and anisotropic confinement in QCD.
%\newblock {\em Physical Review D}, 66:045006, 2002.
%%

%\bibitem{fukushima2012}K. Fukushima, Y. Hidaka. \newblock Magnetic catalysis versus magnetic inhibition.
%%%\newblock {\em Physical review letters}, 110:031601, 2013.
%

\bibitem{alford2004dense}
Mark Alford.
\newblock Dense quark matter in nature.
\newblock {\em Progress of Theoretical Physics Supplement}, 153:1--14, 2004.

\bibitem{buballa2005njl}
Michael Buballa.
\newblock Njl-model analysis of dense quark matter.
\newblock {\em Physics Reports}, 407(4):205--376, 2005.

\bibitem{alford1998qcd}
Mark Alford, Krishna Rajagopal, and Frank Wilczek.
\newblock Qcd at finite baryon density: Nucleon droplets and color
  superconductivity.
\newblock {\em Physics Letters B}, 422(1):247--256, 1998.

\bibitem{alford2000magnetic}
Mark Alford, J{\"u}rgen Berges, and Krishna Rajagopal.
\newblock Magnetic fields within color superconducting neutron star cores.
\newblock {\em Nuclear Physics B}, 571(1):269--284, 2000.

\bibitem{ferrer2005magnetic}
Efrain~J Ferrer, Vivian de~La~Incera, and Cristina Manuel.
\newblock Magnetic color-flavor locking phase in high-density qcd.
\newblock {\em Physical review letters}, 95(15):152002, 2005.

\bibitem{fukushima2008color}
Kenji Fukushima and Harmen~J Warringa.
\newblock Color superconducting matter in a magnetic field.
\newblock {\em Physical review letters}, 100(3):032007, 2008.

\bibitem{ferrer2006color}
Efrain~J Ferrer, Vivian de~la Incera, and Cristina Manuel.
\newblock Color-superconducting gap in the presence of a magnetic field.
\newblock {\em Nuclear Physics B}, 747(1):88--112, 2006.

\bibitem{ferrer2007magnetic}
Efrain~J Ferrer and Vivian de~La~Incera.
\newblock Magnetic phases in three-flavor color superconductivity.
\newblock {\em Physical Review D}, 76(4):045011, 2007.

%%\bibitem{allen2015magnetized}P~Allen, Ana~G Grunfeld, and Norberto~N Scoccola.
%\newblock Magnetized color superconducting cold quark matter within the su (2)  f njl model: A novel regularization scheme.
%\newblock {\em Physical Review D}, 92(7):074041, 2015.

\bibitem{sen2015anisotropic}
Srimoyee Sen.
\newblock Anisotropic propagator for the goldstone modes in color-flavor locked
  phase in the presence of a magnetic field.
\newblock {\em Physical Review D}, 92(2):025004, 2015.

\bibitem{giannakis2002ginzburg}
Ioannis Giannakis and Hai-cang Ren.
\newblock Ginzburg-landau free energy functional of color superconductivity at
  weak coupling.
\newblock {\em Physical Review D}, 65(5):054017, 2002.

\bibitem{iida2002superfluid}
Kei Iida and Gordon Baym.
\newblock Superfluid phases of quark matter. iii. supercurrents and vortices.
\newblock {\em Physical Review D}, 66(1):014015, 2002.

\bibitem{balachandran2006semisuperfluid}
A P~Balachandran, S~Digal, and T~Matsuura.
\newblock Semisuperfluid strings in high density qcd.
\newblock {\em Physical Review D}, 73(7):074009, 2006.

\bibitem{nakano2008non}
Eiji Nakano, Muneto Nitta, and Taeko Matsuura.
\newblock Non-abelian strings in high-density qcd: Zero modes and interactions.
\newblock {\em Physical Review D}, 78(4):045002, 2008.

\bibitem{eto2014vortices}
Minoru Eto, Yuji Hirono, Muneto Nitta, and Shigehiro Yasui.
\newblock Vortices and other topological solitons in dense quark matter.
\newblock {\em Progress of Theoretical and Experimental Physics},
  2014(1):012D01, 2014.

\bibitem{zhang2015magnetic}
Xiao-Bing Zhang, Zhi-Cheng Bu, Fu-Ping Peng, and Yi~Zhang.
\newblock Magnetic effects on color--flavor-locked quark matter and non-abelian
  vortices via ginzburg--landau approach.
\newblock {\em Nuclear Physics A}, 938:1--13, 2015.

\bibitem{vinci2012spontaneous}
Walter Vinci, Mattia Cipriani, and Muneto Nitta.
\newblock Spontaneous magnetization through non-abelian vortex formation in
  rotating dense quark matter.
\newblock {\em Physical Review D}, 86(8):085018, 2012.

\bibitem{eto2010instabilities}
Minoru Eto, Muneto Nitta, and Naoki Yamamoto.
\newblock Instabilities of non-abelian vortices in dense qcd.
\newblock {\em Physical review letters}, 104(16):161601, 2010.

\bibitem{eto2009color}
Minoru Eto and Muneto Nitta.
\newblock Color magnetic flux tubes in dense qcd.
\newblock {\em Physical Review D}, 80(12):125007, 2009.

\bibitem{vilenkin2000cosmic}
Alexander Vilenkin and E~Paul~S Shellard.
\newblock {\em Cosmic strings and other topological defects}.
\newblock Cambridge University Press, 2000.

\bibitem{witten1985superconducting}
Edward Witten.
\newblock Superconducting strings.
\newblock {\em Nuclear Physics B}, 249(4):557--592, 1985.

\bibitem{davis1988physics1}
RL~Davis and EPS Shellard.
\newblock The physics of vortex superconductivity.
\newblock {\em Physics Letters B}, 207(4):404--410, 1988.

\bibitem{davis1988physics2}
RL~Davis and E~Paul~S Shellard.
\newblock The physics of vortex superconductivity. ii.
\newblock {\em Physics Letters B}, 209(4):485--490, 1988.

\bibitem{haws1988superconducting}
David Haws, Mark Hindmarsh, and Neil Turok.
\newblock Superconducting strings or springs?
\newblock {\em Physics Letters B}, 209(2-3):255--261, 1988.

\bibitem{kaplan2002charged}
David~B Kaplan and Sanjay Reddy.
\newblock Charged and superconducting vortices in dense quark matter.
\newblock {\em Physical review letters}, 88(13):132302, 2002.

\bibitem{buckley2002superconducting}
Kirk~BW Buckley and Ariel~R Zhitnitsky.
\newblock Superconducting k strings in high density qcd.
\newblock {\em Journal of High Energy Physics}, 2002(08):013, 2002.

\bibitem{lemperiere2003behaviour}
Y~Lemperiere and EPS Shellard.
\newblock On the behaviour and stability of superconducting currents.
\newblock {\em Nuclear Physics B}, 649(3):511--525, 2003.

\bibitem{bedaque2011vortons}
Paulo~F Bedaque, Evan Berkowitz, and Aleksey Cherman.
\newblock Vortons in dense quark matter.
\newblock {\em Physical Review D}, 84(2):023006, 2011.

\bibitem{iida2005magnetic}
Kei Iida.
\newblock Magnetic vortex in color-flavor locked quark matter.
\newblock {\em Physical Review D}, 71(5):054011, 2005.

\bibitem{ferrer2006magnetic}
Efrain~J Ferrer and Vivian de~La~Incera.
\newblock Magnetic fields boosted by gluon vortices in color superconductivity.
\newblock {\em Physical review letters}, 97(12):122301, 2006.

\bibitem{Ritus1985Quantum}
Ritus, V. I.
\newblock Quantum effects of the interaction of elementary particles with an intense electromagnetic field.
\newblock{\em Journal of Soviet Laser Research}, 6(5):497-617, 1985.


\bibitem{Elizalde2004Neutrino}
Elizalde, E. and Ferrer, E. J. and Incera, V. De La.
\newblock Neutrino Propagation in a Strongly Magnetized Medium.
\newblock {\em Physical Review D}, 70(4):636-640, 2004.




\end{thebibliography}


\end{document}
